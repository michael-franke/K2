\documentclass[fleqn,reqno,10pt]{article}



%========================================
% Packages
%========================================

\usepackage[]{mypackages}

\bibliography{MyRefGlobal}

\usepackage[]{mathdesign}
\usepackage{MnSymbol}
\usepackage{ulsy}
\usepackage{wasysym}
\usepackage{ascii}
\usepackage{marvosym}
\usepackage{pifont}
\usepackage{phaistos}
\usepackage{epsdice}
\usepackage{skak}




%========================================
% Theorem Environments
%========================================

\usepackage{myenvironments}



%========================================
% Commands
%========================================

\usepackage{mycommand_local}


%========================================
% Page Layout
%========================================
% \renewcommand{\baselinestretch}{1.1} % multiplies vertical distance between lines in a paragraph
% \setlength{\parskip}{1ex plus 0.4ex minus 0.1ex}
% \addtolength{\voffset}{-1.7cm} \addtolength{\hoffset}{-1.5cm}
% \addtolength{\textwidth}{3.1cm} \addtolength{\textheight}{2.3cm}
% \addtolength{\voffset}{-1cm} \addtolength{\hoffset}{-0.5cm}
% \addtolength{\textwidth}{1cm} \addtolength{\textheight}{2cm}
% \setlength{\parindent}{0cm} % set paragraph indentation

%========================================
% More Layout
%========================================

% Itemize
\renewcommand{\labelitemi}{\large{$\mathbf{\cdot}$}}    % itemize symbols
\renewcommand{\labelitemii}{\large{$\mathbf{\cdot}$}}
\renewcommand{\labelitemiii}{\large{$\mathbf{\cdot}$}}
\renewcommand{\labelitemiv}{\large{$\mathbf{\cdot}$}}

% Description
\renewcommand{\descriptionlabel}[1]{\hspace\labelsep\textsc{#1}}

% Figure Captions
\usepackage{caption} % use corresponding myfiguresize!
\setlength{\captionmargin}{20pt}
\renewcommand{\captionfont}{\small}
\setlength{\belowcaptionskip}{7pt} % standard is 0pt

% Equationnumbering
%\numberwithin{equation}{section}

% sectioning
\usepackage{sectsty}
\chapterfont{\rmfamily\mdseries\Large\raggedleft}
\sectionfont{\rmfamily\mdseries\large\nohang\centering}
\subsectionfont{\rmfamily\mdseries\normalsize\nohang\centering}
\subsubsectionfont{\rmfamily\mdseries\normalsize\itshape\nohang\centering}
\paragraphfont{\rmfamily\mdseries\normalsize\textsc}

% for indentation of section headings:
%\sectionfont{\hspace*{-2.3em}\rmfamily\mdseries\bfseries\large\nohang\raggedright}


%========================================
% Document
%========================================


\title{Sentences for use in Experiment 2}

\date{February 25, 2011 }

\begin{document}

\maketitle

\section{Target Sentences}

\begin{itemize}
\item target sentences have a scalar item embedded under a quantifier
  \begin{itemize}
  \item embedding quantifiers:
    \begin{enumerate}
    \item ``alle'' \attrib{(upward monotonic)}
    \item ``genau ein'' \attrib{(non-monotonic)}
    \end{enumerate}
  \item two scalar items in embedded position:
    \begin{enumerate}
    \item ``einige''
    \item ``oder''
    \end{enumerate}
  \end{itemize}
  \begin{exe}
  \ex \label{bsp:target1} Alle der \dots sind  mit einigen ihrer \dots
    verbunden. \attrib{($\forall\exists$)}
  \ex Genau einer der \dots ist  mit einigen seiner \dots verbunden. \attrib{($!\exists\exists$)}
  \ex Alle der \dots sind  mit ihrem \dots oder \dots verbunden. \attrib{($\forall\vee$)}
  \ex Genau einer der \dots ist  mit seinem \dots oder \dots verbunden. \attrib{($!\exists\vee$)}
  \end{exe}
\item each target sentence should occur in three versions, e.g.:
  \begin{exe}
    \exr{bsp:target1} Alle der \dots sind mit einigen ihrer \dots
    verbunden. \attrib{($\forall\exists$)}
    \begin{xlist}
    \ex  Alle der \dots sind mit einigen ihrer \dots
      verbunden. \attrib{(flat)}
    \ex Alle der \dots sind mit \emph{einigen} ihrer \dots
      verbunden. \attrib{(stress)}
    \ex Alle der \dots sind nur mit \emph{einigen} ihrer \dots
      verbunden. \attrib{(only + stress)}
    \end{xlist}
  \end{exe}
\item we test for 3 possible readings of each target sentence
\end{itemize}

\section{Controls \& Fillers}

\begin{itemize}
\item to check if \acros{vp}\ understand ``incremental information
  evaluation'' in general
\end{itemize}

\medskip

\begin{exe}
  \ex Genau zwei der \dots sind mit mindestens einem der \dots verbunden.
  \ex Mindestens zwei der \dots sind mit mindestens eimem der \dots verbunden.
\end{exe}

\begin{itemize}
\item to check if \acros{vp}\ behave consistently with genuinely
  ambiguous sentences
  \begin{itemize}
  \item potential problem: disambiguation by intonation
  \end{itemize}
\end{itemize}

\medskip

\begin{exe}
  \ex There are triangles and squares that are connected with all of
    their circles.
  \ex There are blue triangles and squares.
\end{exe}


\section{Fillers}

\begin{exe}
  \ex Alle Quadrate sind rechts von ihrem Stern.
  \ex Der Stern ist links neben dem Quadrat.
  \ex Alle blauen Sterne sind unverbunden.
  \ex Genau ein Stern ist zwischen seinem dem Quadrat und seinem
    Dreieck. 
\end{exe}

\newpage

\appendix

\section{Target Sentences}
\label{sec:target-sentences}

\subsection{$\forall\exists$}
\label{sec:forallexists}

\begin{exe}
  \ex
    \begin{xlist}
      \ex Alle der Häkchen ($\checkmark$) sind mit einigen ihrer Kreise verbunden.
      \ex Alle der Kreuze ($\maltese$) sind mit einigen ihrer Vierecke
        verbunden.
      \ex Alle der Sternchen ($\bigstar$) sind mit einigen ihrer
        Dreiecke verbunden.
      \ex Alle der Dreiecke ($\filledmedtriangleup$) sind mit einigen ihrer
        Kreise verbunden.
      \ex Alle der Punkte ($\medbullet$) sind mit einigen ihrer
        Vierecke verbunden.
      \ex Alle der Pfeile ($\text{\ding{220}}$) sind mit einigen ihrer
        Dreiecke verbunden.
      \ex Alle der Punkte ($\filledmedsquare$) sind mit einigen ihrer
        Vierecke verbunden.
      \ex Alle der Pentagramme ($\pentagram$) sind mit einigen ihrer
        Kreise verbunden.
      \ex Alle der Blitze ($\text{\blitze}$) sind mit einigen ihrer
        Dreiecke verbunden.
      \ex Alle der Fragezeichen ($\text{?}$) sind mit einigen ihrer
        Dreiecke verbunden.
      \ex Alle der Herzchen ($\varheartsuit$) sind mit einigen ihrer
        Kreise verbunden.
      \ex Alle der Picksymbole ($\spadesuit$) sind mit einigen ihrer
        Kreise verbunden.
    \end{xlist}
\end{exe}

\subsection{$!\exists\exists$}

\begin{exe}
\ex
  \begin{xlist}
  \ex Genau eine der Ellipsen ($\partialvartoiint$) ist mit einigen ihrer
    Vierecke verbunden.     
  \ex Genau einer der Monde ($\leftmoon$) ist mit einigen seiner
    Vierecke verbunden.
    \ex Genau eines der Notenzeichen ($\text{\SO}$) ist mit einigen seiner
    Vierecke verbunden.
    \ex Genau einer der Smileys ($\text{\SOH}$) ist mit einigen seiner
    Vierecke verbunden.
    \ex Genau eine der Sonnen ($\text{\SI}$) ist mit einigen seiner
    Vierecke verbunden.
    \ex Genau eines der Karos ($\text{\EOT}$) ist mit einigen seiner
    Vierecke verbunden.
    \ex Genau eines der Telefone ($\text{\Telefon}$) ist mit einigen seiner
    Vierecke verbunden.
    \ex Briefe (\Letter)
    \ex Handys (\Mobilefone)
    \ex Gefahrsymbole (\Biohazard)
    \ex Stoppschilder (\Stopsign)
    \ex Scheren (\Rightscissors)
  \end{xlist}

\end{exe}

\subsection{$\forall\vee$}

\begin{exe}
\ex
  \begin{xlist}
    \ex Bleistifte (\ding{46})
    \ex Davidsterne (\davidsstar)
    \ex Eiskristalle (\ding{100})
    \ex Blumen (\ding{96})
    \ex Davidsterne (\davidsstar)
    \ex Eiskristalle (\ding{100})
    \ex Blumen (\ding{96})
    \ex Fünfecke (\pentagon)
    \ex Flugzeuge (\ding{40})
    \ex Balken (\ding{121})
    \ex Halbkreis (\ding{119})
    \ex Taube (\PHdove)
  \end{xlist}
\end{exe}

\subsection{$!\exists\vee$}

\begin{exe}
\ex
  \begin{xlist}
  \ex Punker (\PHplumedHead)
  \ex Fisch (\PHtunny)
  \ex Männchen (\PHchild)
  \ex Äste (\PHplaneTree)
  \ex Böcke (\PHram)
  \ex Blätter (\textleaf)
  \ex Uhren (\clock)
  \ex Glocken (\bell)
  \ex Fußbälle (\Football)
  \ex Fahrräder (\Bicycle)
  \ex Kaffetassen (\Coffeecup)
  \ex Fabriken (\Industry)
  \end{xlist}
\end{exe}

\subsection{Controls}

\begin{exe}
\ex
  \begin{xlist}
  \ex Bügeleisen (\IroningIII)
  \ex Yingyangsymbole (\Yinyang)
  \ex Fledermäuse (\Bat)
  \ex Anarchiesymbole (\CircledA)
  \ex Würfel (\epsdice{5})
  \ex Springer (\symknight)
  \ex Türme (\symrook)
  \end{xlist}
\end{exe}




\subsection{Filler}






% \printbibliography[heading=bibintoc]

\end{document}
