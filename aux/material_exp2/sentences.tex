\documentclass[fleqn,reqno,10pt]{article}



%========================================
% Packages
%========================================

\usepackage[]{mypackages}

\bibliography{MyRefGlobal}

%\usepackage[garamond]{mathdesign}




%========================================
% Theorem Environments
%========================================

\usepackage{myenvironments}


%========================================
% Commands
%========================================

\usepackage{mycommands}


%========================================
% Page Layout
%========================================
% \renewcommand{\baselinestretch}{1.1} % multiplies vertical distance between lines in a paragraph
% \setlength{\parskip}{1ex plus 0.4ex minus 0.1ex}
% \addtolength{\voffset}{-1.7cm} \addtolength{\hoffset}{-1.5cm}
% \addtolength{\textwidth}{3.1cm} \addtolength{\textheight}{2.3cm}
% \addtolength{\voffset}{-1cm} \addtolength{\hoffset}{-0.5cm}
% \addtolength{\textwidth}{1cm} \addtolength{\textheight}{2cm}
% \setlength{\parindent}{0cm} % set paragraph indentation

%========================================
% More Layout
%========================================

% Itemize
\renewcommand{\labelitemi}{\large{$\mathbf{\cdot}$}}    % itemize symbols
\renewcommand{\labelitemii}{\large{$\mathbf{\cdot}$}}
\renewcommand{\labelitemiii}{\large{$\mathbf{\cdot}$}}
\renewcommand{\labelitemiv}{\large{$\mathbf{\cdot}$}}

% Description
\renewcommand{\descriptionlabel}[1]{\hspace\labelsep\textsc{#1}}

% Figure Captions
\usepackage{caption} % use corresponding myfiguresize!
\setlength{\captionmargin}{20pt}
\renewcommand{\captionfont}{\small}
\setlength{\belowcaptionskip}{7pt} % standard is 0pt

% Equationnumbering
%\numberwithin{equation}{section}

% sectioning
\usepackage{sectsty}
\chapterfont{\rmfamily\mdseries\Large\raggedleft}
\sectionfont{\rmfamily\mdseries\large\nohang\centering}
\subsectionfont{\rmfamily\mdseries\normalsize\nohang\centering}
\subsubsectionfont{\rmfamily\mdseries\normalsize\itshape\nohang\centering}
\paragraphfont{\rmfamily\mdseries\normalsize\textsc}

% for indentation of section headings:
%\sectionfont{\hspace*{-2.3em}\rmfamily\mdseries\bfseries\large\nohang\raggedright}


%========================================
% Document
%========================================


\title{Sentences for use in Experiment 2}

\date{February 25, 2011 }

\begin{document}

\maketitle

\section{Target Sentences}

\begin{itemize}
\item target sentences have a scalar item embedded under a quantifier
  \begin{itemize}
  \item embedding quantifiers:
    \begin{enumerate}
    \item definite article \attrib{(for baseline)}
    \item ``einige'' \attrib{(upward monotonic)}
    \item ``alle'' \attrib{(upward monotonic)}
    \item ``genau ein'' \attrib{(non-monotonic)}
    \end{enumerate}
  \item two scalar items in embedded position:
    \begin{enumerate}
    \item ``einige''
    \item ``oder''
    \end{enumerate}
  \end{itemize}
  \begin{exe}
  \ex \label{bsp:target1} Der \dots ist mit einigen seiner \dots
    verbunden. \attrib{($\iota\exists$)}
  \ex Einige der \dots sind  mit einigen ihrer \dots
    verbunden. \attrib{($\exists\exists$)}
  \ex Alle der \dots sind  mit einigen ihrer \dots verbunden. \attrib{($\forall\exists$)}
  \ex Genau einer der \dots ist  mit einigen seiner \dots verbunden. \attrib{($!\exists\exists$)}
  \ex Der \dots ist mit seinem \dots oder \dots verbunden. \attrib{($\iota\vee$)}
  \ex Einige der \dots sind  mit ihrem \dots oder \dots verbunden. \attrib{($\exists\vee$)}
  \ex Alle der \dots sind  mit ihrem \dots oder \dots verbunden. \attrib{($\forall\vee$)}
  \ex Genau einer der \dots ist  mit seinem \dots oder \dots verbunden. \attrib{($!\exists\vee$)}
  \end{exe}
\item each target sentence should occur in four (or three?) versions,
  e.g.:
  \begin{exe}
  \exr{bsp:target1} Der \dots ist mit einigen seiner \dots
    verbunden. \attrib{($\iota\exists$)}
    \begin{xlist}
    \ex Der \dots ist mit einigen seiner \dots
      verbunden. \attrib{(flat)}
    \ex Der \dots ist mit \emph{einigen} seiner \dots
      verbunden. \attrib{(stress)}
    \ex Der \dots ist nur mit einigen seiner \dots
      verbunden. \attrib{(only, flat)}
    \ex Der \dots ist nur mit \emph{einigen} seiner \dots
      verbunden. \attrib{(only, stress)}
    \end{xlist}
  \end{exe}

\end{itemize}

\section{Controls \& Fillers}

\begin{itemize}
\item to check if \acros{vp}\ understand ``incremental information
  evaluation'' in general
\end{itemize}

\medskip

\begin{exe}
  \ex Genau zwei der \dots sind mit mindestens einem der \dots verbunden.
  \ex Mindestens zwei der \dots sind mit mindestens eimem der \dots verbunden.
\end{exe}

\begin{itemize}
\item to check if \acros{vp}\ behave consistently with genuinely
  ambiguous sentences
  \begin{itemize}
  \item potential problem: disambiguation by intonation
  \end{itemize}
\end{itemize}

\medskip

\begin{exe}
  \ex There are triangles and squares that are connected with all of
    their circles.
  \ex There are blue triangles and squares.
\end{exe}


\section{Fillers}

\begin{exe}
  \ex Alle Quadrate sind rechts von ihrem Stern.
  \ex Der Stern ist links neben dem Quadrat.
  \ex Alle blauen Sterne sind unverbunden.
  \ex Genau ein Stern ist zwischen seinem dem Quadrat und seinem
    Dreieck. 
\end{exe}



% \printbibliography[heading=bibintoc]

\end{document}
