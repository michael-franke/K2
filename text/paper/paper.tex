\documentclass[fleqn,reqno,10pt]{article}

\usepackage{myarticlestyledefault}


\usepackage{mypackages}
\usepackage{mycommands}
\usepackage{myenvironments}

\usepackage[]{svninfo}



\newcommand{\lit}{\acro{lit}}
\newcommand{\glb}{\acro{glb}}
\newcommand{\loc}{\acro{loc}}

\newcommand{\as}{\acro{as}}
\renewcommand{\es}{\acro{es}}
\renewcommand{\AE}{\as}
\newcommand{\GE}{\es}

\newcommand{\lc}{\acro{lc}}
\newcommand{\ec}{\acro{ec}}

\DefineNamedColor{named}{mycol}{cmyk}{0.6,0.6,0,0}
\DefineNamedColor{named}{mygray}{cmyk}{0.05,0.05,0.05,0.05}
\DefineNamedColor{named}{mygraylight}{cmyk}{0.017,0.017,0.017,0.017}

\newcommand{\mymark}[1]{{\color{mycol}{#1}}}

\title{Scalar Items in Embedded Position: {A}nother Experimental Approach}
\author{Fabian Schlotterbeck, Michael Franke and Petra Augurzky}
\date{}

\begin{document}
\maketitle


\begin{abstract}
  \dots make concrete \dots
\end{abstract}

\svnInfo $Id$

\section{Introduction}
\label{sec:introduction}

This paper deals with the interpretation of two types of
sentences, exemplified by:

\begin{exe}
\ex \label{bsp:AE}
  \mymark{All} of the students read \mymark{some} of the
  papers. \hfill (\AE)
\ex \label{bsp:GE} 
  \mymark{Exactly one} of the students read \mymark{some} of the
  papers. \hfill (\GE)
\end{exe}

\noindent In \as-sentences a scalar item \emph{some} takes scope under
a universal quantifier \emph{all}. In \es-sentences a scalar item
\emph{some} takes scope under a non-monotonic quantifier \emph{exactly
  one}. Scalar \emph{some} is usually assumed to receive a semantic
interpretation similar to logical $\exists$, so that the expression
\emph{Some boys cried.} is literally true in a situation where all
boys cried. But it is also known to invite upper-bounding inferences
in plain utterances:

\begin{exe}
  \ex \label{bsp:Plain-SI}
    \begin{xlist}
      \ex \label{bsp:Plain-SI-Target} Hans solved some of the problems.
      \ex \label{bsp:Plain-SI-Implicature} $\implicates$ Hans solved some but not all of the problems.
    \end{xlist}
\end{exe}

\noindent The classical explanation of this inference, following the
pioneering work of \citet{Grice1975:Logic-and-Conve}, has it that
(\ref{bsp:Plain-SI-Implicature}) is a pragmatic inference, a so-called
\emph{quantity implicature}, derived by an abductive inference as the
best explanation of why am informed, knowledgable and cooperative
speaker has uttered (\ref{bsp:Plain-SI-Target}) when she could also
have uttered the semantically stronger and relevant:\dn{redo labels of
  examples}

\begin{exe}
  \exr{bsp:Plain-SI}
    \begin{xlist}
      \exi{c.} \label{bsp:Plain-SI-Alternative} Hans solved all of the problems.
    \end{xlist}
\end{exe}

\noindent If this upper-bounding inference would occur often and
systematically enough, then it may well be that also embedded
occurrences of \emph{some} get enriched, in some fashion or other, to
contribute \emph{in situ} the enriched meaning \emph{some but not
  all}. Indeed, even \citeauthor{Grice1975:Logic-and-Conve} envisaged
this possibility when he wrote: ``It may not be impossible for what
starts life, so to speak, as a conversational implicature to become
conventionalized'' \citep[p.58]{Grice1975:Logic-and-Conve}. But then
there are at least three relevant candidate readings for \as- and
\es-sentences (see Section~\ref{sec:get-know-your} for more detail and
some helpful illustrations): (i) a \emph{literal reading} like in
(\ref{bsp:AE-Literal}) and (\ref{bsp:GE-Literal}) where \emph{some}
has only its literal meaning as in; (ii) a \emph{global reading} like
in (\ref{bsp:AE-Literal}) and (\ref{bsp:GE-Literal}) where, according
to the Gricean intuition, we compare utterances of (\ref{bsp:AE}) or
(\ref{bsp:GE}) with utterances of the corresponding sentences
(\ref{bsp:AE-Alternative}) and (\ref{bsp:GE-Alternative}) where
\emph{some} is replaced by \emph{all}; and also (iii) a \emph{local
  reading} like in (\ref{bsp:AE-Local}) and (\ref{bsp:GE-Local}) where
\emph{some} is read to mean \emph{some but not all} in the scope of
the embedding quantifier.


% DIRTY!!!
\setcounter{exx}{0}

\begin{exe}

  \ex \mymark{All} of the students read {\mymark{some}} of the
  papers. 

  \begin{xlist}
  \ex \label{bsp:AE-Literal} \mymark{All} of the students read
    {\mymark{some and maybe all}} of the papers. \hfill (\as-\lit)
  \ex \label{bsp:AE-Global}
    \mymark{All} of the students read \mymark{some} 
    and  \hfill (\as-\glb)\\
    \mymark{not all} of the students read \mymark{all} of the papers.
  \ex \label{bsp:AE-Local}
    \mymark{All} of the students read {\mymark{some  but not all}} of the
    papers. \hfill (\as-\loc)
  \end{xlist}

\ex \mymark{Exactly one} of the students read {\mymark{some}} of the
  papers.

  \begin{xlist}
  \ex \label{bsp:GE-Literal} \mymark{Exactly one} of the students read
    {\mymark{some and maybe all}} of the papers. \hfill (\es-\lit)
  \ex \label{bsp:GE-Global}
    \mymark{Exactly one} of the students read \mymark{some} 
    and  \hfill (\es-\glb)\\
    \mymark{not all} of the students read \mymark{all} of the papers.
  \ex \label{bsp:GE-Local}
    \mymark{Exactly one} of the students read {\mymark{some  but not all}} of the
    papers. \hfill (\es-\loc)
  \end{xlist}

\ex \label{bsp:AE-Alternative} \mymark{All} of the students read
  {\mymark{all}} of the papers. 

\ex \label{bsp:GE-Alternative} \mymark{Exactly one} of the students
  read {\mymark{all}} of the papers.

\end{exe}

The main question we are interested in here is which of these three
conceivable readings is available to subjects na\"{i}ve to all
pragmatic theory. We are moreover interested in the more refined
question which of the available readings na\"{i}ve subjects prefer.

Addressing these issues empirically is highly relevant because they
lie at the heart of the current debate of the exact location and
nature of the interface between semantics and pragmatics. In rough
approximation, there are three camps ---more or less fiercely---
involved in this border war. (We will give a more detailed description
of the various theoretical positions below in
Section~\ref{sec:theories-predictions}.) Firstly, there are
\emph{pragmatic traditionalists} who seek to conserve the spirit of
\citeauthor{Grice1975:Logic-and-Conve}'s
(\citeyear{Grice1975:Logic-and-Conve}) original ideas as much as
possible
\citep[e.g.][]{Spector2006:Scalar-Implicat,Sauerland2004:Scalar-Implicat,Russell2006:Against-Grammat,vanRooijSchulz:ExhaustiveInterpretation,Geurts2010:Quantity-Implic,Franke2011:Quantity-Implic}. Traditionalists
happily acknowledge the existence of global readings, but might
consider local readings either unavailable or a beast distinct from
scalar implicatures. Opposed to that is the camp of \emph{lexical
  conventionalists}
\citep[e.g.][]{LevinsonPresumptiveMeanings2000,Chierchia:2004_ScalarImplicatures}
who maintain a double lexical meaning for scalar \emph{some}, one as
in logical semantics, and one with the enriched meaning \emph{some but
  not all}. As the latter is considered a default, lexical
conventionalism has no problem accounting for local readings, and in
fact would consider these the preferred readings. Thirdly and finally,
there is also the ever more powerful camp of \emph{grammaticalists}
who defend that the distribution of upper-bounded readings of
\emph{some} is best explained by postulating a silent operator, akin
to the meaning of the particle \emph{only} \citep{Chierchia2006:Broaden-Your-Vi,Fox2007:Free-Choice-and,ChierchiaFox2008:The-Grammatical,Chierchia2012:FC-Nominals-and,Sauerland2012:The-Computation}. This operator may be
applied in compositional semantics also in the scope of other logical
operators, so as to account for, e.g., local readings of \as- and
\es-sentences.





\bigskip





\section{Get to know your readings}
\label{sec:get-know-your}

\section{Theories and predictions}
\label{sec:theories-predictions}

\section{Previous studies}
\label{sec:previous-studies}





\printbibliography[heading=bibintoc]

\end{document}
