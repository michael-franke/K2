\documentclass[a4paper,10pt]{article}


\usepackage{t1enc}
\usepackage{linguex}
\usepackage{harvard}
\usepackage{graphicx}
\usepackage{amsmath}
\usepackage{amssymb}
\usepackage{graphicx}
\usepackage{tikz}
\usetikzlibrary{arrows,automata}
\usepackage{subfigure}
\usepackage{qtree}
%\usepackage{hyperref}




\bibliographystyle{diss}

%opening
\title{Critique of previous studies and motivation}
\author{}



\newtheorem{hypothese}{Hypothesis}
\newtheorem{Def}{Definition}
\newtheorem{Prop}{Proposition}
\newtheorem{Example}{Example}

\begin{document}

\maketitle

\begin{abstract}

\end{abstract}

\section{Critique of previous studies and motivation}

Summing up, the mentioned studies have presented diverging evidence concerning the availability of local implicatures. Varoius problematic aspects of these studies have already been discussed elsewhere (C\&D, C\&S, van Tiel). Here, we want to point out two particular methodological problems that do impede their interpretation. However, we also want to make clear that these problems follow almost directly from the nature of the phenomenon under investigation and therefore seem nearly impossible to avoid.   

Firstly,  a mapping between responses to be obtained in an experiment and readings of the sentences is problematic due to the logical dependencies between the relevant readings discussed in section \ref{xyz}. It is impossible to construct situations that clearly correspond to local readings of AS-sentences, since the local reading of such sentences implies their global and local readings. Every situation which is consitent with such a sentence under its local reading is also consitent with the other two readings. Therefore, we cannot test all the cases we might wish to. It would be especially desirable to test situations that are consitent with the local reading, but inconsitent with the global and literal reading of AS-sentences, in a picture verification task. In that case 'yes'-responses would unambiguously indicate local readings, whereas 'no'-responses would indicate their absence. However, since this is impossible we are left with suboptimal choices. In the experiment of Geurts situations that are inconsitent with local readings but consitent with the other two readings were tested. In that case 'no'-responses would be decisive~-- when compared to a suitable baseline. However, 'yes' responses were observed exclusively. These can be attributed to one of the other readings. Therefore, assuming a principle of charity (Davidson 2001) to be at play in picture verification tasks, the data of Geurts do not provide evidence against the existence of local readings.  In  the experiments of C\&S and C\&D on the other hand situations that are consitent with all three readings were tested. Here again 'yes'-judgments cannot be interpreted confidently because they can always be attributed to global or literal readings. Both C\&S and C\&D attempt to surpass this problem by interpreting relative preferences. This does, however, bring us to the second problem.

Preferences may be affectd by graphical properties of the picture materials. For example, verification complexity or typicality of the pictures with respect to the test sentences may differ between conditions and thereby induce certain preferences. In his careful critique van Tiel has eleborated on this point. Van Tiel points out that the relative preferences C\&S and C\&D interpret can be explained solely in terms of typicality, because the {\bf all}-pictures (see e.g. Figure \ref{xx}) are more typical instances of the sentence meaning than the competing pictures, irrespective of local implicatures.

To conclude, in order to decide about the availablity of local implicatures we would ideally like to test situations that can unambiguously be mapped to the different readings and do not differ in typcality or other crucial properties. Given the logical dependencies inherent to the relevant readings and the considerations concerning typicality presented by van Tiel this seems impossible to achieve. As we willl argue below we do, however, believe it is.      



\bibliography{../../Literatur/tex/literatur}
\end{document}
