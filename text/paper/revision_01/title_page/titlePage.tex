\documentclass[fleqn,reqno,12pt]{article}

%========================================
% Packages
%========================================

\usepackage[]{mypackages}
\bibliography{MyRefGlobal}
\usepackage{mycommands}

\usepackage{array}
% \usepackage{authblk}

\newcommand{\ia}{\acro{ia}}


\title{Embedded Scalars, Preferred Readings \& Intonation: {A}n
  Experimental Revisit} 
\author{Michael Franke, Fabian Schlotterbeck and Petra Augurzky
} 

\date{}



\begin{document}
\maketitle

\thispagestyle{empty}

\vspace*{-1cm}
\begin{center}
  Universtit\"at T\"ubingen \\
  Seminar f\"ur Sprachwissenschaft \\
  Wilhelmstra\ss e 19\\
  72072 T\"ubingen\\
  Germany  \\
  \bigskip \href{mailto:mchfranke@gmail.com}{corresponding email: \tt
    mchfranke@gmail.com}
\end{center}

 \vspace*{1cm}



\begin{abstract}
  The scalar item \emph{some} is widely assumed to receive a meaning
  enrichment to \emph{some but not all} if it occurs in matrix
  position. The question under which circumstances this enrichment can
  occur in certain embedded positions plays an important role in
  deciding how to delineate semantics and pragmatics. % Some claim
  % that it requires marked intonation for the relevant pragmatic
  % enrichment to occur in embedded position; others allow for it more
  % freely.
  We present new experimental data that bear on this theoretical
  issue. In distinction to previous experimental approaches, we
  presented sentence material auditorily in order to explicitly
  control prosodic markedness of the scalar item. Moreover, our
  experiment sheds light on the relative preferences or salience of
  candidate readings. The presented data turn out to be unexpected
  under a traditional Gricean view, but also suggest refinement of the
  more recent grammatical approach.
\end{abstract}


\end{document}
