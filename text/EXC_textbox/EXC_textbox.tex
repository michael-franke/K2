\documentclass[fleqn,reqno,10pt]{article}



%========================================
% Packages
%========================================

\usepackage[]{mypackages}

%\usepackage[natbib=true,style=authoryear-comp,backend=bibtex,doi=false,url=false]{biblatex}

\bibliography{MyRefGlobal}

%\usepackage[garamond]{mathdesign}




%========================================
% Theorem Environments
%========================================

\usepackage{myenvironments}


%========================================
% Commands
%========================================

\usepackage{mycommands}
%========================================
% Page Layout
%========================================
% \renewcommand{\baselinestretch}{1.1} % multiplies vertical distance between lines in a paragraph
% \setlength{\parskip}{1ex plus 0.4ex minus 0.1ex}
% \addtolength{\voffset}{-1.7cm} \addtolength{\hoffset}{-1.5cm}
% \addtolength{\textwidth}{3.1cm} \addtolength{\textheight}{2.3cm}
% \addtolength{\voffset}{-1cm} \addtolength{\hoffset}{-0.5cm}
% \addtolength{\textwidth}{1cm} \addtolength{\textheight}{2cm}
% \setlength{\parindent}{0cm} % set paragraph indentation

%========================================
% More Layout
%========================================

% Itemize
\renewcommand{\labelitemi}{\large{$\mathbf{\cdot}$}}    % itemize symbols
\renewcommand{\labelitemii}{\large{$\mathbf{\cdot}$}}
\renewcommand{\labelitemiii}{\large{$\mathbf{\cdot}$}}
\renewcommand{\labelitemiv}{\large{$\mathbf{\cdot}$}}

% Description
\renewcommand{\descriptionlabel}[1]{\hspace\labelsep\textsc{#1}}

% Figure Captions
\usepackage{caption} % use corresponding myfiguresize!
\setlength{\captionmargin}{20pt}
\renewcommand{\captionfont}{\small}
\setlength{\belowcaptionskip}{7pt} % standard is 0pt

% Equationnumbering
%\numberwithin{equation}{section}

% sectioning
\usepackage{sectsty}
\chapterfont{\rmfamily\mdseries\Large\raggedleft}
\sectionfont{\rmfamily\mdseries\large\nohang\centering}
\subsectionfont{\rmfamily\mdseries\normalsize\nohang\centering}
\subsubsectionfont{\rmfamily\mdseries\normalsize\itshape\nohang\centering}
\paragraphfont{\rmfamily\mdseries\normalsize\textsc}

% for indentation of section headings:
%\sectionfont{\hspace*{-2.3em}\rmfamily\mdseries\bfseries\large\nohang\raggedright}


%========================================
% Document
%========================================


\title{Processing of Scalar Items in Embedded Positions}

\author{Petra Augurzky, Michael Franke \& Fabian Schlotterbeck}
\date{}

\begin{document}

\maketitle

\textbf{Scalar implicatures} are inferences such as from
(\ref{bsp:SI_Kiki_Target}) to (\ref{bsp:SI_Kiki_Implicature}). These
inference hinge on comparing what the speaker \emph{actually} said
with what she \emph{could have said}. Roughly speaking, since the
determiner ``all'' is a salient alternative to the determiner ``some''
that was used in (\ref{bsp:SI_Kiki_Target}), the inference in
(\ref{bsp:SI_Kiki_Implicature}) is licensed.

\noindent \begin{minipage}{0.5\linewidth}
  \begin{exe}
  \ex  \label{bsp:SI_Kiki_Target} Some of my friends are
    bald.
  \end{exe}
\end{minipage}
\begin{minipage}{0.5\linewidth}
  \begin{exe}
  \ex  \label{bsp:SI_Kiki_Implicature} Not all of my friends are
    bald.
  \end{exe}
\end{minipage}

\noindent Recently there has been renewed interest in the question
what inferences arise from utterances where scalar items like ``some''
occur in embedded position
\citep[cf.][]{Cohen1971:Some-Remarks-on,Landman1998:Plurals-and-Max,Chierchia:2004_ScalarImplicatures,Magri2009:A-Theory-of-Ind}. For
example, a sentence like (\ref{bsp:AllSome}a) could get a \textbf{global
enrichment} as in (\ref{bsp:AllSome_Implicature_Global}) by comparing it to the whole alternative utterance in
(\ref{bsp:AllSome}b), or it could be strengthened to a reading as in
(\ref{bsp:AllSome}c) where the scalar item ``some'' gets a \textbf{local
enrichment} in the scope of another logical operator. 
\noindent \begin{minipage}{0.4\linewidth}
\begin{exe}
\ex \label{bsp:AllSome} All of the students read
\end{exe}
\end{minipage}
\begin{minipage}{0.3\linewidth}
  a. some of the books.
\end{minipage}
\begin{minipage}{0.3\linewidth}
  b. all of the books.
\end{minipage}
\noindent \begin{minipage}{0.4\linewidth}
 $ $ 
\end{minipage}
\begin{minipage}{0.6\linewidth}
  c. some but not all of the books.
\end{minipage}
\begin{minipage}{1.0\linewidth}
  \begin{exe}
  \ex \label{bsp:AllSome_Implicature_Global} All read some and some did
    not read all. 
  \end{exe}
\end{minipage}

\noindent Empirical evidence on the availability of either global and
local enrichments in relevant constructions is sparse and, where it
exists, not unanimous
\citep{GeurtsPouscoulous2009:Embedded-Implic,ChemlaSpector2010:Experimental-Ev}.
The reason is that global and local enrichments are not locally
independent and therefore standard picture verification tasks run into
difficulties because we cannot keep the properties of the visual
stimuli constant across to-be-compared conditions. In cooperation
between the \acro{sfb} and the \acro{sfs}, we therefore designed an
amended picture verification task where subjects had to sequentially
unveil visual information if necessary for a truth-value judgement. We
also studied \textbf{the effects of prosody on pragmatic enrichments}
and found that global enrichments are hardly attested, while local
enrichments occur seldom with neutral prosody, but more frequent if an
embedded scalar items is intonationally marked. In future work we
intend to extend the developed paradigm beyond embedding quantifiers
``all'' and ``exactly one'' to other quantifiers and also modal
expressions and to probe more deeply into the effect prosody on
pragmatic interpretation.


\printbibliography[heading=bibintoc]

\end{document}
