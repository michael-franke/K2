\documentclass[fleqn,reqno,10pt]{article}

\usepackage{myarticlestyledefault}


\usepackage{subfig}
\usepackage{todonotes}

\newcommand{\lit}{\acro{lit}}
\newcommand{\glb}{\acro{glb}}
\newcommand{\loc}{\acro{loc}}

\newcommand{\as}{\acro{as}}
\renewcommand{\es}{\acro{es}}
\renewcommand{\AE}{\as}
\newcommand{\GE}{\es}
\newcommand{\ea}{\acro{ea}}

\newcommand{\lc}{\acro{lc}}
\newcommand{\ec}{\acro{ec}}
\newcommand{\LC}{\lc}
\newcommand{\EC}{\ec}


\title{K2 Analysis}
\author{}
\date{}


\begin{document}
\maketitle


\section{Target Conditions}
\label{sec:target-conditions}

There were two critical sentence-sequence types, \as and \es, which
are the levels in factor \textsc{Condition}. Factor \textsc{Accent}
codes whether a sentence was neutral or accented. Every participant
saw each critical sentence type six times, of which three times had
neutral, three times constrastive stress intonation on the scalar
item. We coded into factor \textsc{Position} whether a particular
encounter of a critical sentence and intonation pattern was the first,
second or third time for a participant. Finally, we coded judgements
as either \emph{literal}, \emph{local}, \emph{global} or \emph{false}
for form factor \emph{Response}. Given these factors, contingency
tables of the observed frequencies are in
Table~\ref{tab:CC-ContingencyTable}.
%
\begin{table}[h]

\subfloat[][\as neutral]{\begin{tabular}{lcccc}
    & \emph{literal} & \emph{local} & \emph{global} & \emph{false} \\
    \midrule
    Pos 1 & 29 &  7 & 0 & 1 \\
    Pos 2 & 26 & 10 & 0 & 1 \\
    Pos 3 & 25 &  9 & 1 & 2 \\ \addlinespace[0.15cm]
    Total & 80 & 26 & 1 & 4
\end{tabular}}
%
\hspace{0.2cm}
%
\subfloat[][\es neutral]{\begin{tabular}{lcccc}
    & \emph{literal} & \emph{local} & \emph{global} & \emph{false} \\
    \midrule
    Pos 1 & 16 &  7 & 0 & 14 \\
    Pos 2 & 21 & 13 & 0 &  3 \\
    Pos 3 & 16 & 13 & 0 &  8 \\ \addlinespace[0.15cm]
    Total & 53 & 33 & 0 & 25
\end{tabular}}

\subfloat[][\as accented]{\begin{tabular}{lcccc}
    & \emph{literal} & \emph{local} & \emph{global} & \emph{false} \\
    \midrule
    Pos 1 & 28 &  7 & 1 & 1 \\
    Pos 2 & 25 &  8 & 1 & 3 \\
    Pos 3 & 24 & 10 & 0 & 3 \\ \addlinespace[0.15cm]
    Total & 77 & 25 & 2 & 7
\end{tabular}}
%
\hspace{0.2cm}
%
\subfloat[][\es accented]{\begin{tabular}{lcccc}
    & \emph{literal} & \emph{local} & \emph{global} & \emph{false} \\
    \midrule
    Pos 1 & 13 &  6 & 0 & 18 \\
    Pos 2 & 24 &  7 & 1 &  5\\
    Pos 3 & 19 & 13 & 0 &  5 \\ \addlinespace[0.15cm]
    Total & 56 & 26 & 1 & 28
\end{tabular}}

\caption{Contingency table of raw judgements for target condition.}
\label{tab:CC-ContingencyTable}
\end{table}

Two things strike the eye. Firstly, there are hardly any \emph{global}
responses. For subsequent analysis in terms of log-linear models we
must therefore lump together \emph{global} and \emph{false} responses
into one more general category level \emph{other}. Secondly, the
number of \emph{false} responses drops remarkably in the
\es-conditions. The following analysis should probe into this
potential position effect.




To test which factors are relevant to explain the data distribution in
Table~\ref{tab:CC-ContingencyTable}, we compute log-linear models
\citep{KnokeBurke1980:Log-Linear-Mode}. The relevant factors are
\textsc{Position}, \textsc{Response}, \textsc{Condition}, and
\textsc{Accent}. To ensure applicability, we lump levels \emph{global}
and \emph{false} of factor \textsc{Response} into a single level
\emph{other}. The best predictor of the data takes all factors
\textsc{Position}, \textsc{Response} and \textsc{Condition} and all
their interactions, including the three-way interaction, into account
($\chi^2 = 7.493$, $df=18$, $p = 0.985$, $AIC = 43.493$ (compared to
$AIC=72$ of the saturated model)). Crucially, factor \textsc{Accent}
is spurious.

% To test whether the effect of \textsc{Condition} could be due to the
% relatively high number of \emph{false} responses in the
% \es-conditions, we regrouped the factor \textsc{Response} once more
% to distinguish only \emph{local} responses from \emph{other}
% responses. Doing so allows us to address the question whether the
% amount of \emph{local} responses differed across conditions and
% accentuation types. But this does not seem to be the case. The model
% that predicts the data best in this case contains factors
% \textsc{Response} and \textsc{Position} and their two-way
% interaction ($\chi^2 = 4.71$, $df=18$, $p = 0.999$, $AIC = 16.711$
% (compared to $AIC=48$ of the saturated model)). This suggests that a
% shift in the distribution of \emph{local} responses can be
% satisfactorily explained as a position effect that is constant
% across conditions: later trials yielded less \emph{other} responses.

To test, then, whether \textsc{Position} affects the distribution of
\emph{literal} and \emph{local} readings, we also computed the best
log-linear predictor for the subset of our data including only
\emph{literal} and \emph{local} responses. Doing so, the best
predictor indeed drops factor \textsc{Position} and contains only
factors \textsc{Response} and \textsc{Condition} and their two-way
interaction ($\chi^2 = 12.497$, $df=20$, $p = 0.898$, $AIC = 20.497$
(compared to $AIC=48$ of the saturated model)). This suggests that the
factor \textsc{Position} is not relevant for explaining the
distribution of \emph{literal} and \emph{local} readings. Its effects
noted previously are most likely due to the decrease of \emph{false}
responses in later trials of the \es-conditions.


\section{Preference-Related Controls}
\label{sec:pref-relat-contr}

Some subjects saw three others four trials of preference-related
controls. The relevant factors in this case are: \textsc{Condition}
with levels \emph{lc} and \emph{ec} coding which reading could be
judged first in a sequence; \textsc{Accent} with levels \emph{ntr},
\emph{lc} and \emph{ec} coding which reading was cued by intonation;
\textsc{Position} with levels 1 to 4; and \textsc{Response} with
levels \emph{lc}, \emph{ec} and \emph{false}. The observed data under
this classification is given in Table~\ref{tab:TF-ContingencyTable}.
%
\begin{table}[h]

 \centering

\subfloat[][\lc neutral]{\begin{tabular}{lccc}
    & \emph{lc} & \emph{ec} & \emph{false} \\
    \midrule
    Pos 1 & 30 &  6 & 1  \\
    Pos 2 & 24 &  9 & 4  \\
    Pos 3 & 25 &  8 & 3  \\
    Pos 4 & 11 & 10 & 1  \\ \addlinespace[0.15cm]
    Total & 90 & 33 & 9 
\end{tabular}}
%
\hspace{0.4cm}
%
\subfloat[][\ec neutral]{\begin{tabular}{lccc}
    & \emph{lc} & \emph{ec} & \emph{false} \\
    \midrule
    Pos 1 &  28 &   4 &   5 \\ 
    Pos 2 &  26 &  10 &   1 \\ 
    Pos 3 &  19 &  11 &   6 \\ 
    Pos 4 &  11 &   9 &   2 \\ \addlinespace[0.15cm]
    Total &  65 &  34 &  14 
\end{tabular}}

\subfloat[][\lc with \lc cue]{\begin{tabular}{lccc}
    & \emph{lc} & \emph{ec} & \emph{false} \\
    \midrule
    Pos 1 &  26 &   6 &   5 \\ 
    Pos 2 &  30 &   4 &   3 \\ 
    Pos 3 &  30 &   5 &   2 \\ 
    Pos 4 &  14 &   1 &   1 \\ 
    \addlinespace[0.15cm]
    Total & 100 & 16 & 11 
\end{tabular}}
%
\hspace{0.4cm}
%
\subfloat[][\ec with \lc cue]{\begin{tabular}{lccc}
    & \emph{lc} & \emph{ec} & \emph{false} \\
    \midrule
    Pos 1 &  24 &   6 &   7 \\ 
    Pos 2 &  30 &   3 &   4 \\ 
    Pos 3 &  30 &   5 &   2 \\ 
    Pos 4 &  13 &   2 &   1 \\ 
    \addlinespace[0.15cm]
    Total &  97 &  16 &  14 
\end{tabular}}


\subfloat[][\lc with \ec cue]{\begin{tabular}{lccc}
    & \emph{lc} & \emph{ec} & \emph{false} \\
    \midrule
    Pos 1 &  15 &  15 &   7 \\ 
    Pos 2 &  20 &  10 &   7 \\ 
    Pos 3 &  24 &   7 &   5 \\ 
    Pos 4 &   8 &   8 &   6 \\
    \addlinespace[0.15cm]
    Total &  67 & 40 &  25 
\end{tabular}}
%
\hspace{0.4cm}
%
\subfloat[][\ec with \ec cue]{\begin{tabular}{lccc}
    & \emph{lc} & \emph{ec} & \emph{false} \\
    \midrule
    Pos 1 &  21 &  12 &   4 \\ 
    Pos 2 &  18 &  12 &   7 \\ 
    Pos 3 &  22 &  12 &   2 \\ 
    Pos 4 &  13 &   7 &   2 \\ 
    \addlinespace[0.15cm]
    Total &  74 &  43 &  15 
\end{tabular}}

\caption{Contingency table of raw judgements for target-related fillers.}
\label{tab:TF-ContingencyTable}
\end{table}

The best hierarchical log-linear model (by $AIC$) includes factors
\textsc{Response}, \textsc{Accent} and \textsc{Position}, as well as
the two-way interaction between \textsc{Response} and \textsc{Accent}
($\chi^2 = 48.802$, $df=60$, $p=0.85$, $AIC=72.802$ (compared to
$AIC=144$ of the standard model)). Crucially, the observed response
pattern does not differ significantly across conditions (i.e., which
reading can be judged first), but does depend on intonational cues.

The factor \textsc{Position} cannot be dropped without dramatically
reducing explanatory power of the model. However, if we abstract over
intonation and conditions, we get the following table:
\begin{center}
\centering
\begin{tabular}{rrrr}
  & \emph{lc} & \emph{ec} & \emph{false} \\ 
  \midrule
  Pos 1 & 144 & 49 & 29 \\ 
  Pos 2 & 148 & 48 & 26 \\ 
  Pos 3 & 150 & 48 & 20 \\ 
  Pos 4 &  70 & 37 & 13 \\ 
\end{tabular}
\end{center}
There is no interaction between \textsc{Position} and
\textsc{Response} ($\chi^2$, $df=6$, $p=0.3798$), suggesting that
\textsc{Position} does not affect the overal distribution of readings.

\newpage

\section{Loose Ends}
\label{sec:loose-ends}

If we only look at the judgement that subjects gave to the very first
target condition that they saw (be it \as or \es), we obtain the
following table:
\begin{center}
\begin{tabular}{lcccc}
    & \emph{literal} & \emph{local} & \emph{global} & \emph{false} \\
    \midrule
    accented & 15 &  5 & 0 &  9 \\
    neutral  &  6 &  0 & 0 &  2 \\ \addlinespace[0.15cm]
    Total    & 21 &  5 & 0 & 11
\end{tabular}
\end{center}
Notably, nobody gave a \emph{local} answer on the first encounter of a
target condition with neutral accentuation. Still, Pearson's $\chi^2$
test does not reject the null hypothesis of an equal distribution of
readings ($\chi^2 = 2.0546$, $p \approx 0.4318$).  \todo[inline]{Is
  this test applicable at all? Don't we have too little data points?
  How else could a dependence be tested in this case?}

\printbibliography[heading=bibintoc]

\end{document}
