\documentclass[fleqn,reqno,10pt]{article}

\usepackage{myarticlestyledefault}


\usepackage{subfig}
\usepackage{todonotes}

\title{K2 Analysis}
\author{Michael Franke}
\date{}

\newcommand{\lit}{\acro{lit}}
\newcommand{\glb}{\acro{glb}}
\newcommand{\loc}{\acro{loc}}

\newcommand{\as}{\acro{as}}
\renewcommand{\es}{\acro{es}}
\renewcommand{\AE}{\as}
\newcommand{\GE}{\es}
\newcommand{\ea}{\acro{ea}}

\newcommand{\lc}{\acro{lc}}
\newcommand{\ec}{\acro{ec}}
\newcommand{\LC}{\lc}
\newcommand{\EC}{\ec}

\begin{document}
\maketitle

\section{Controls}
\label{sec:controls}



\section{Target Conditions}
\label{sec:target-conditions}

Each participant saw each critical sentence type three times with
neutral and three times with constrastive stress intonation on the
scalar item. We coded judgements as either \emph{literal},
\emph{local}, \emph{global} or \emph{false}. Contingency tables of the
observed frequencies are in Table~\ref{tab:CC-ContingencyTable}.
%
\begin{table}[h]

\subfloat[][\as neutral]{\begin{tabular}{lcccc}
    & \emph{literal} & \emph{local} & \emph{global} & \emph{false} \\
    \midrule
    Pos 1 & 29 &  7 & 0 & 1 \\
    Pos 2 & 26 & 10 & 0 & 1 \\
    Pos 3 & 25 &  9 & 1 & 2 \\ \addlinespace[0.15cm]
    Total & 80 & 26 & 1 & 4
\end{tabular}}
%
\hspace{0.2cm}
%
\subfloat[][\es neutral]{\begin{tabular}{lcccc}
    & \emph{literal} & \emph{local} & \emph{global} & \emph{false} \\
    \midrule
    Pos 1 & 16 &  7 & 0 & 14 \\
    Pos 2 & 21 & 13 & 0 &  3 \\
    Pos 3 & 16 & 13 & 0 &  8 \\ \addlinespace[0.15cm]
    Total & 53 & 33 & 0 & 25
\end{tabular}}

\subfloat[][\as accented]{\begin{tabular}{lcccc}
    & \emph{literal} & \emph{local} & \emph{global} & \emph{false} \\
    \midrule
    Pos 1 & 28 &  7 & 1 & 1 \\
    Pos 2 & 25 &  8 & 1 & 3 \\
    Pos 3 & 24 & 10 & 0 & 3 \\ \addlinespace[0.15cm]
    Total & 77 & 25 & 2 & 7
\end{tabular}}
%
\hspace{0.2cm}
%
\subfloat[][\es accented]{\begin{tabular}{lcccc}
    & \emph{literal} & \emph{local} & \emph{global} & \emph{false} \\
    \midrule
    Pos 1 & 13 &  6 & 0 & 18 \\
    Pos 2 & 24 &  7 & 1 &  5\\
    Pos 3 & 19 & 13 & 0 &  5 \\ \addlinespace[0.15cm]
    Total & 56 & 26 & 1 & 28
\end{tabular}}

\caption{Contingency table of raw judgements.}
\label{tab:CC-ContingencyTable}
\end{table}
%
Two things strike the eye. Firstly, there are hardly any \emph{global}
responses. For subsequent analysis in terms of log-linear models we
must therefore lump together \emph{global} and \emph{false} responses
into one more general category \emph{other}. Secondly, the number of
\emph{false} responses drops remarkably in the \es-conditions. The
following analysis should probe into this potential position effect.


If we only look at the very first judgement that subjects gave to a
target condition, we obtain the following table:
\begin{center}
\begin{tabular}{lcccc}
    & \emph{literal} & \emph{local} & \emph{global} & \emph{false} \\
    \midrule
    accented & 15 &  5 & 0 &  9 \\
    neutral  &  6 &  0 & 0 &  2 \\ \addlinespace[0.15cm]
    Total    & 21 &  5 & 0 & 11
\end{tabular}
\end{center}
Notably, nobody gave a \emph{local} answer on the first encounter of a
target condition with neutral accentuation. Still, Pearson's $\chi^2$
test does not reject the null hypothesis of an equal distribution
of readings ($\chi^2 = 2.0546$, $p \approx 0.4318$).
\todo[inline]{Is this test applicable at all? Don't we have too little
data points? How else could a dependence be tested in this case?}

To test which factors are relevant to explain the data distribution in
Table~\ref{tab:CC-ContingencyTable}, we compute log-linear models
\citep{KnokeBurke1980:Log-Linear-Mode}. The relevant factors are
\textsc{Position}, \textsc{Response}, \textsc{Condition}, and
\textsc{Accent}. To ensure applicability, we lump levels \emph{global}
and \emph{false} of factor \textsc{Response} into a single level
\emph{other}. The best predictor of the data takes all factors
\textsc{Position}, \textsc{Response} and \textsc{Condition} and all
their interactions into account ($\chi^2 = 7.493$, $df=18$, $p =
0.985$, $AIC = 43.493$ (compared to $AIC=72$ of the saturated
model)). Crucially, factor \textsc{Accent} is spurious.

To test whether the effect of \textsc{Condition} could be due to the
relatively high number of \emph{false} responses in the
\es-conditions, we regrouped the factor \textsc{Response} once more to
distinguish only \emph{local} responses from \emph{other}
responses. Doing so allows us to address the question whether the
amount of \emph{local} responses differed across conditions and
accentuation types. But this does not seem to be the case. The
smallest model that predicts the data well in this case contains
factors \textsc{Response} and \textsc{Position} and their two-way
interaction. 



\section{Preference-Related Controls}
\label{sec:pref-relat-contr}




\printbibliography[heading=bibintoc]

\end{document}
