\documentclass[fleqn,reqno,10pt]{article}

\usepackage{myarticlestyledefault}


\usepackage{subfig}
\usepackage{todonotes}

\newcommand{\lit}{\acro{lit}}
\newcommand{\glb}{\acro{glb}}
\newcommand{\loc}{\acro{loc}}

\newcommand{\as}{\acro{as}}
\renewcommand{\es}{\acro{es}}
\renewcommand{\AE}{\as}
\newcommand{\GE}{\es}
\newcommand{\ea}{\acro{ea}}

\newcommand{\lc}{\acro{lc}}
\newcommand{\ec}{\acro{ec}}
\newcommand{\LC}{\lc}
\newcommand{\EC}{\ec}


\title{K2 Analysis}
\author{}
\date{}


\begin{document}
\maketitle


\section{Target Conditions}
\label{sec:target-conditions}

We excluded two participants from the analysis due to poor performance
on control conditions (VPs 2 and 29).

There were two critical sentence-sequence types, \as and \es, which
are the levels in factor \textsc{Condition}. Factor \textsc{Accent}
codes whether a sentence was neutral or accented. Every participant
saw each critical sentence type six times, of which three times had
neutral, three times constrastive stress intonation on the scalar
item. We coded into factor \textsc{Position} whether a particular
encounter of a critical sentence and intonation pattern was the first,
second or third time for a participant. Finally, we coded judgements
as either \emph{literal}, \emph{local}, \emph{global} or \emph{false}
for form factor \emph{Response}. Given these factors, contingency
tables of the observed frequencies are in
Table~\ref{tab:CC-ContingencyTable}.
%
\begin{table}[h]

\subfloat[][\as neutral]{\begin{tabular}{lcccc}
    & \emph{literal} & \emph{local} & \emph{global} & \emph{false} \\
    \midrule
    Pos 1 & 35 &  0 & 0 & 3 \\
    Pos 2 & 32 &  3 & 0 & 3 \\
    Pos 3 & 33 &  3 & 0 & 2 \\ \addlinespace[0.15cm]
    Total &100 &  6 & 0 & 8
\end{tabular}}
%
\hspace{0.2cm}
%
\subfloat[][\es neutral]{\begin{tabular}{lcccc}
    & \emph{literal} & \emph{local} & \emph{global} & \emph{false} \\
    \midrule
    Pos 1 & 19 &  5 & 0 & 14 \\
    Pos 2 & 25 &  2 & 0 & 11 \\
    Pos 3 & 21 &  6 & 0 & 11 \\ \addlinespace[0.15cm]
    Total & 65 & 13 & 0 & 36
\end{tabular}}

\subfloat[][\as accented]{\begin{tabular}{lcccc}
    & \emph{literal} & \emph{local} & \emph{global} & \emph{false} \\
    \midrule
    Pos 1 & 33 &  2 & 1 &  2 \\
    Pos 2 & 31 &  4 & 1 &  2 \\
    Pos 3 & 31 &  5 & 0 &  2 \\ \addlinespace[0.15cm]
    Total & 95 & 11 & 2 &  6
\end{tabular}}
%
\hspace{0.2cm}
%
\subfloat[][\es accented]{\begin{tabular}{lcccc}
    & \emph{literal} & \emph{local} & \emph{global} & \emph{false} \\
    \midrule
    Pos 1 & 24 &  8 & 0 &  6 \\
    Pos 2 & 22 &  7 & 0 &  9 \\
    Pos 3 & 27 &  6 & 0 &  5 \\ \addlinespace[0.15cm]
    Total & 73 & 21 & 0 & 20
\end{tabular}}

\caption{Contingency table of raw judgements for target condition.}
\label{tab:CC-ContingencyTable}
\end{table}

To test which factors are relevant to explain the data distribution in
Table~\ref{tab:CC-ContingencyTable}, we compute log-linear models
\citep{KnokeBurke1980:Log-Linear-Mode}. The relevant factors are
\textsc{Position}, \textsc{Response}, \textsc{Condition}, and
\textsc{Accent}. As there are hardly any \emph{global} responses, we
must lump levels \emph{global} and \emph{false} of factor
\textsc{Response} into a single level \emph{other}, because otherwise
log-linear regression is not applicable. We select the ``best''
predictor of the data by a gradient search over hierarchical models in
terms of AICs. The best model takes factors \textsc{Accent},
\textsc{Response} and \textsc{Condition} into account, as well as the
interaction terms \textsc{Response:Accent} and
\textsc{Response:Condition} ($\chi^2 = 14.29$, $df=27$, $p = 0.98$,
$AIC = 32.29$ (compared to $AIC=72$ of the saturated
model)). Crucially, factor \textsc{Position} is spurious. Response
patterns only depended on the type of the sentence and whether it was
produced with accentuation. 

\todo[inline]{ what's the argument to accept that there are local readings
    for neutral accentuation in the first place? ; make plots with ``raw responses'' on position
}




\section{Preference-Related Controls}
\label{sec:pref-relat-contr}

For our preference-related control conditions the relevant factors
are: \textsc{Condition} with levels \emph{lc} and \emph{ec} coding
which reading could be judged first in a sequence, \textsc{Accent}
with levels \emph{ntr}, \emph{lc} and \emph{ec} coding which reading
was cued by intonation, \textsc{Response} with levels \emph{lc},
\emph{ec} and \emph{false}, and finally \textsc{Position} encoding how
frequently subjects had seen a given sentence-accentuation pair
before. Some subjects saw three, others four trials, so that to ensure
unbiased applicability of our log-linear models, we excluded every
fourth encounter of the same sentence type and accentuation. The
observed data under this classification is given in
Table~\ref{tab:TF-ContingencyTable}.
%
\begin{table}[h]

 \centering

\subfloat[][\lc neutral]{\begin{tabular}{lccc}
    & \emph{lc} & \emph{ec} & \emph{false} \\
    \midrule
    Pos 1 & 23 &  9 & 6  \\
    Pos 2 & 23 & 11 & 4  \\
    Pos 3 & 27 &  8 & 3  \\ \addlinespace[0.15cm]
    Total & 73 & 28 & 13 
\end{tabular}}
%
\hspace{0.4cm}
%
\subfloat[][\ec neutral]{\begin{tabular}{lccc}
    & \emph{lc} & \emph{ec} & \emph{false} \\
    \midrule
    Pos 1 &  22 &   5 &  11 \\ 
    Pos 2 &  28 &   5 &   5 \\ 
    Pos 3 &  29 &   7 &   2 \\ \addlinespace[0.15cm]
    Total &  79 &  17 &  18 
\end{tabular}}

\subfloat[][\lc with \lc cue]{\begin{tabular}{lccc}
    & \emph{lc} & \emph{ec} & \emph{false} \\
    \midrule
    Pos 1 &  26 &   6 &   6 \\ 
    Pos 2 &  30 &   5 &   3 \\ 
    Pos 3 &  34 &   2 &   2 \\ 
    \addlinespace[0.15cm]
    Total & 90  & 13 & 11 
\end{tabular}}
%
\hspace{0.4cm}
%
\subfloat[][\ec with \lc cue]{\begin{tabular}{lccc}
    & \emph{lc} & \emph{ec} & \emph{false} \\
    \midrule
    Pos 1 &  30 &   4 &   4 \\ 
    Pos 2 &  33 &   4 &   1 \\ 
    Pos 3 &  31 &   4 &   3 \\ 
    \addlinespace[0.15cm]
    Total &  94 &  12 &  8 
\end{tabular}}


\subfloat[][\lc with \ec cue]{\begin{tabular}{lccc}
    & \emph{lc} & \emph{ec} & \emph{false} \\
    \midrule
    Pos 1 &  15 &  18 &   5 \\ 
    Pos 2 &  16 &  16 &   6 \\ 
    Pos 3 &  15 &  16 &   6 \\
    \addlinespace[0.15cm]
    Total &  46 & 50 &  17 
\end{tabular}}
%
\hspace{0.4cm}
%
\subfloat[][\ec with \ec cue]{\begin{tabular}{lccc}
    & \emph{lc} & \emph{ec} & \emph{false} \\
    \midrule
    Pos 1 &  18 &  12 &   8 \\ 
    Pos 2 &  18 &   9 &  11 \\ 
    Pos 3 &  18 &  10 &  10 \\
    \addlinespace[0.15cm]
    Total &  54 &  31 &  29 
\end{tabular}}

\caption{Contingency table of raw judgements for target-related fillers.}
\label{tab:TF-ContingencyTable}
\end{table}

Just like for the target conditions, the best hierarchical log-linear
model (by $AIC$) includes factors \textsc{Response}, \textsc{Accent}
and \textsc{Condition}, as well as the two-way interactions
\textsc{Response:Accent} and \textsc{Response:Condition} ($\chi^2 =
23.14$, $df=42$, $p=0.99$, $AIC=47.14$ (compared to $AIC=108$ of the
saturated model)). Crucially, the observed response pattern does not
differ significantly across conditions (i.e., which reading can be
judged first), but does depend on intonational cues.

\section{Loose Ends}
\label{sec:loose-ends}

If we only look at the judgement that subjects gave to the very first
target condition that they saw (be it \as or \es), we obtain the
following table:
\begin{center}
\begin{tabular}{lcccc}
    & \emph{literal} & \emph{local} & \emph{global} & \emph{false} \\
    \midrule
    accented & 10 &  3 & 1 &  2 \\
    neutral  & 15 &  0 & 0 &  7 \\ \addlinespace[0.15cm]
    Total    & 25 &  3 & 0 &  9
\end{tabular}
\end{center}
Notably, nobody gave a \emph{local} answer on the first encounter of a
target condition with neutral accentuation. Pearson's $\chi^2$ test
does reject the null hypothesis of an equal distribution of readings
($\chi^2 = 7.0051$, $p \approx 0.034$).

\printbibliography[heading=bibintoc]

\end{document}
