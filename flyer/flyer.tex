\documentclass{article}

\usepackage[palatino]{mypackages}\usepackage{mycommands}

\usepackage{eurosym}

\usepackage{verbatim}
\usepackage[active,tightpage]{preview}
\PreviewEnvironment{tikzpicture}
\setlength\PreviewBorder{5pt}

\begin{document}


\begin{tikzpicture}[rounded corners]

  \node at (0,5.5) (Heading) {{\Huge Versuchspersonen Gesucht!!!}};

  \node[text width = 10cm] at (0,3.5) (Heading) {{\large Für ein
      Experiment zur Sprachverarbeitung suchen wir ab sofort deutsche
      Mutterspachler.}};


  \node[text width = 10cm] at (0,2) (Heading) {{\large Versuchspersonen werden gefragt, ob bestimmte
    Sätze bestimmte Bilder angemessen beschreiben.}};

  \node[text width = 10cm] at (0,1.35) (Heading) {{\large In etwa so:}};

  \node at (0,0) (sentence) {``Genau eine Schere ist mit keiner ihrer blauen Formen
              verbunden.''};

  \node at (0,-5) (graphics) {\includegraphics[scale=0.4]{train_5_6.pdf}};

  \node[draw = black] (wahr) at (-3,-10) {Wahr $\Box$};

  \node[draw = black] (falsch) at (3,-10) {Falsch $\Box$};

    \begin{pgfonlayer}{background}
      \node [draw=black!35,fill=black!7,fit=(wahr) (falsch) (sentence)
      (graphics)] {};
    \end{pgfonlayer}

  \node[text width = 10cm] at (0,-11.5) (Heading) {{\large
      Dauer: ca.~40\,min}};

  \node[text width = 10cm] at (0,-12) (Heading) {{\large
      Ort: Wilhelmstraße 19}};

  \node[text width = 10cm] at (0,-12.5) (Heading) {{\large
      \color{red}{Aufwandsentschädigung: \euro{8}}}};

  \node[text width = 10cm] at (0,-13.5) (Heading) {{\large
      Termine flexibel nach Absprache.}};

  \node[text width = 10cm] at (0,-14) (Heading) {{\large
      Bei Interesse bitte melden unter:}};

  \node[text width = 10cm] at (0,-14.5) (Heading) {{\large
      \url{versucheb1@sfb833.uni-tuebingen.de}}};

\end{tikzpicture}


\end{document}

\begin{frame}
  \begin{block}{Please Join Our Experiment!}
    \begin{itemize}
    \item 8 Euro for ca.~40min
    \item make appointment: \url{versucheb1@sfb833.uni-tuebingen.de}
    \end{itemize}
    \begin{center}
      \framebox{\parbox{0.9\textwidth}{\begin{center} {\footnotesize
              ``Genau eine Schere ist mit keiner ihrer blauen Formen
              verbunden.''} \\
      \includegraphics[height = 0.4\textheight]{train_5_6.pdf}\\
      {\footnotesize Wahr $\Box$ \hspace{4cm} Falsch $\Box$}
    \end{center}}}
\end{center}
